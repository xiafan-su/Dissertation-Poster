%%%%%%%%%%%%%%%%%%%%%%%%%%%%%%%%%%%%%%%%%
% Jacobs Landscape Poster
% LaTeX Template
% Version 1.1 (14/06/14)
%
% Created by:
% Computational Physics and Biophysics Group, Jacobs University
% https://teamwork.jacobs-university.de:8443/confluence/display/CoPandBiG/LaTeX+Poster
%
% Further modified by:
% Nathaniel Johnston (nathaniel@njohnston.ca)
%
% This template has been downloaded from:
% http://www.LaTeXTemplates.com
%
% License:
% CC BY-NC-SA 3.0 (http://creativecommons.org/licenses/by-nc-sa/3.0/)
%
%%%%%%%%%%%%%%%%%%%%%%%%%%%%%%%%%%%%%%%%%

%----------------------------------------------------------------------------------------
%	PACKAGES AND OTHER DOCUMENT CONFIGURATIONS
%----------------------------------------------------------------------------------------

\documentclass[final]{beamer}

\usepackage[scale=1.12]{beamerposter} % Use the beamerposter package for laying out the poster
\usepackage{listings}
\usepackage{fontspec}

\usepackage{multirow}

\setmainfont{Georgia}

\usetheme{confposter} % Use the confposter theme supplied with this template

\setbeamercolor{block title}{fg=ngreen,bg=white} % Colors of the block titles
\setbeamercolor{block body}{fg=black,bg=white} % Colors of the body of blocks
\setbeamercolor{block alerted title}{fg=white,bg=dblue!70} % Colors of the highlighted block titles
\setbeamercolor{block alerted body}{fg=black,bg=dblue!10} % Colors of the body of highlighted blocks
% Many more colors are available for use in beamerthemeconfposter.sty

\setbeamerfont{block body}{series=\rmfamily}
%\setbeamerfont{block body}{series=\fontgeorgia}

%-----------------------------------------------------------
% Define the column widths and overall poster size
% To set effective sepwid, onecolwid and twocolwid values, first choose how many columns you want and how much separation you want between columns
% In this template, the separation width chosen is 0.024 of the paper width and a 4-column layout
% onecolwid should therefore be (1-(# of columns+1)*sepwid)/# of columns e.g. (1-(4+1)*0.024)/4 = 0.22
% Set twocolwid to be (2*onecolwid)+sepwid = 0.464
% Set threecolwid to be (3*onecolwid)+2*sepwid = 0.708

\newlength{\sepwid}
\newlength{\onecolwid}
\newlength{\twocolwid}
\newlength{\threecolwid}
\setlength{\paperwidth}{36in} % A0 width: 46.8in
\setlength{\paperheight}{36in} % A0 height: 33.1in
\setlength{\sepwid}{0.024\paperwidth} % Separation width (white space) between columns
\setlength{\onecolwid}{0.22\paperwidth} % Width of one column
\setlength{\twocolwid}{0.464\paperwidth} % Width of two columns
\setlength{\threecolwid}{0.708\paperwidth} % Width of three columns
\setlength{\topmargin}{-0.5in} % Reduce the top margin size
%-----------------------------------------------------------

\usepackage{graphicx}  % Required for including images

\usepackage{booktabs} % Top and bottom rules for tables

%----------------------------------------------------------------------------------------
%	TITLE SECTION
%----------------------------------------------------------------------------------------

\title{Efficient Parser and Pretty Printer Combinators in F2J} % Poster title

\author{Yuteng Zhong, Yi Li and Fan Xia} % Author(s)

\institute{Department of Computer Science, The University of Hong Kong} % Institution(s)

%----------------------------------------------------------------------------------------

\begin{document}

\addtobeamertemplate{block end}{}{\vspace*{2ex}} % White space under blocks
\addtobeamertemplate{block alerted end}{}{\vspace*{2ex}} % White space under highlighted (alert) blocks

\setlength{\belowcaptionskip}{1ex} % White space under figures
\setlength\belowdisplayshortskip{2ex} % White space under equations

\renewcommand{\raggedright}{\leftskip=0pt \rightskip=0pt plus 0cm}

\defverbatim[colored]\makeset{
\begin{lstlisting}


data PList[A]=Nil
|Cons A PList[A];
let rec recursive[A]
(a:A):A=recursive[A]
a;recursive[Int]1

data PList[A] = Nil | Cons A (PList[A]); let rec recursive[A] (a : A) : A = recursive[A] a; recursive[Int] 1

data PList[A]= 	Nil
            |   Cons A PList[A]
;
let rec recursive[A] (a: A): A =
    recursive[A] a
;
recursive[Int] 1

\end{lstlisting}
}


\begin{frame}[t] % The whole poster is enclosed in one beamer frame

\begin{columns}[t] % The whole poster consists of three major columns, the second of which is split into two columns twice - the [t] option aligns each column's content to the top

\begin{column}{\sepwid}\end{column} % Empty spacer column

\begin{column}{\onecolwid} % The first column

%----------------------------------------------------------------------------------------
%	The Tower of Babel
%----------------------------------------------------------------------------------------

\begin{figure}
\includegraphics[width=\linewidth]{img/babellong.png}

\end{figure}

%----------------------------------------------------------------------------------------
%	MOTIVATION
%----------------------------------------------------------------------------------------

\begin{block}{Motivation}

At HKU, the Programming Languages group is developing a new JVM-based compiler for functional languages called F2J.

\begin{figure}
\includegraphics[width=0.9\linewidth]{img/bootstrapping.jpg}

\end{figure}

In order to make a selfhosting compiler, in other word, a \textbf{bootstrapped compiler} for F2J, we will need a parser. In this process, we could also build a pretty printer, which is a reversed process of parsing.

\end{block}

%----------------------------------------------------------------------------------------
%	INTRODUCTION
%----------------------------------------------------------------------------------------
\begin{block}{Introduction}

%A parser is a software component that takes input data and builds a data structure. Then pretty printer takes the data structure and generates designed output.

\textbf{Parsing and Printing:}

\begin{figure}
\includegraphics[width=0.75\linewidth]{img/parseprintershort.png}
\end{figure}


\end{block}


\end{column} % End of the first column

\begin{column}{\sepwid}\end{column} % Empty spacer column

\begin{column}{\twocolwid} % Begin a column which is two columns wide (column 2)

\begin{columns}[t,totalwidth=\twocolwid] % Split up the two columns wide column

\begin{column}{\onecolwid}\vspace{-.6in} % The first column within column 2 (column 2.1)

%----------------------------------------------------------------------------------------
%	MATERIALS
%----------------------------------------------------------------------------------------

\begin{block}{Introduction}

To parse a snippet of specific text, two common ways are used: \\
\textbf{Code generators:}
\begin{itemize}
\item flex
\item yacc
\end{itemize}
\textbf{Parser combinators:}
\begin{itemize}
\item planck in OCaml
\item parsec in Haskell
\end{itemize}

\begin{table}
\centering
\begin{tabular}{c|c|c}
\toprule[4pt]
\textbf{Parsing Type}& \textbf{Pros}& \textbf{Cons}\\
\midrule
\multirow{2}*{Generator}& Pervasive& Heavy config\\
& Mature& Non-scalable\\
\midrule
\multirow{3}*{Combinator}& Efficient& \\
& Modular& Write code\\
& Maintainable& \\
\bottomrule[4pt]
\end{tabular}
\end{table}

%Parser combinators are a set of higher-order functions that accepts several parser as input and then return a parser as output. In this context, a parser is a %function accepting strings as input and returning some structure as output, typically a parse tree or a set of indices representing locations in the string where %parsing stopped successfully. Parser combinators enable a recursive descent parsing strategy that facilitates modular piecewise construction and testing.
%This parsing technique is called combinatory parsing. Parser and Pretty Printer combinators are an alternative to tools used in compiler constructions, such as %lex, yacc or antlr. They have the advantage of being a library instead of a code generator tool.

\end{block}

%----------------------------------------------------------------------------------------

\end{column} % End of column 2.1

\begin{column}{\onecolwid}\vspace{-.6in} % The second column within column 2 (column 2.2)

%----------------------------------------------------------------------------------------
%	METHODS
%----------------------------------------------------------------------------------------

\begin{block}{Methods}

\textbf{Monadic:}

\begin{itemize}

\item Monad: an algebraic structure for machematics
\item Monadic sequencing combinators of parsers
\begin{itemize}
\item Avoids manipulation of nested tuples of results in early work
\item Makes parser more compat and easy to read
\item One monadic parser -> Two simpler monads
\end{itemize}
\item Easy to change

\end{itemize}

% Monad is an algebraic structure from machematics that provided useful for addressing a
% number of computational problems. Using a monadic sequencing combinator of parsers avoids
% the messy manipulation of nested tuples of results present in early work. Moreover, using
% monad comprehension notation makes parsers more compat and easy to read.

\textbf{Packrat:}

Packrat parsing provides the simplicity, elegance and generality of the backtracking model but eliminates the risk of super-linear parse time, by saving all intermediate parsing results as they are computed and ensuring that no result is evaluated more than once.

% The monad of parser can be expressed in a modular way in terms of two simpler monads. The
% immediate benefit is that the basic parser combinators no longer needs to be defined
% explicitly. Rather, they arise automatically as a special case of lifting monad operations
% from the base monad \textit{m} to a certain other monad parameterised over \textit{m}. This
% also means that, if we change the nature of parsers by modifying the base monad, then new
% combinators for the modified monad of parsers also arise automically via the lifting
% construction.


\end{block}

%----------------------------------------------------------------------------------------

\end{column} % End of column 2.2

\end{columns} % End of the split of column 2 - any content after this will now take up 2 columns width

%----------------------------------------------------------------------------------------
%	IMPORTANT RESULT
%----------------------------------------------------------------------------------------
\begin{alertblock}{MONAD}

\begin{figure}
\includegraphics[width=0.8\linewidth]{img/monad.jpg}

\end{figure}
\end{alertblock}

%----------------------------------------------------------------------------------------

\begin{columns}[t,totalwidth=\twocolwid] % Split up the two columns wide column again

\begin{column}{\onecolwid} % The first column within column 2 (column 2.1)

%----------------------------------------------------------------------------------------
%	MATHEMATICAL SECTION
%----------------------------------------------------------------------------------------

\begin{block}{Optimisation}

\textbf{Parser}
\begin{itemize}
\item \textbf{Specific rules:} Try the most possible rules based on parsing context will help to find the correct result earlier.
%The parser will try the rules one by one.
\item \textbf{Lazy evaluation:} Parser combinators will generates all possible results in each level
of parse tree, lazy evaluation can enable the possibilty of ``parse by need''.
%, which will save lots of memory and time and become much more efficient.
\item \textbf{Reduce backtracking:} Backtracking is required when the parser encounter a failure.
% and then it may come back to the begining and try the other alternative rules.

\end{itemize}

\end{block}

%----------------------------------------------------------------------------------------

\end{column} % End of column 2.1

\begin{column}{\onecolwid} % The second column within column 2 (column 2.2)

%----------------------------------------------------------------------------------------
%	RESULTS
%----------------------------------------------------------------------------------------

\begin{block}{Optimisation}



\textbf{Pretty Printer}
\begin{itemize}
\item \textbf{Algebra:} Everything is based on a single concatenation operator that is associative.
\item \textbf{Optimality:} Optimal and Bounded. It means that the lib can choose line breaks to avoid overflow whenever possible, and it's done in O(n).
\item \textbf{Expressiveness:} Not as expressive as Hugh's library, but it is complete and enough for using.

\end{itemize}




%\begin{table}
%\vspace{2ex}
%\begin{tabular}{l l l}
%\toprule
%\textbf{Treatments} & \textbf{Response 1} & \textbf{Response 2}\\
%\midrule
%Treatment 1 & 0.0003262 & 0.562 \\
%Treatment 2 & 0.0015681 & 0.910 \\
%Treatment 3 & 0.0009271 & 0.296 \\
%\bottomrule
%\end{tabular}
%\caption{Table caption}
%\end{table}

\end{block}

%----------------------------------------------------------------------------------------

\end{column} % End of column 2.2

\end{columns} % End of the split of column 2

\end{column} % End of the second column

\begin{column}{\sepwid}\end{column} % Empty spacer column

\begin{column}{\onecolwid} % The third column


\begin{block}{Case Study}

\textbf{Parser and Pretty Printer for F2J}

\begin{figure}
\includegraphics[width=\linewidth]{img/cs.jpg}

\end{figure}



\end{block}

%----------------------------------------------------------------------------------------
%	CONCLUSION
%----------------------------------------------------------------------------------------


\begin{block}{Conclusion}

 The greatest homage is imitation.


\end{block}



%----------------------------------------------------------------------------------------
%	ACKNOWLEDGEMENTS
%----------------------------------------------------------------------------------------

\setbeamercolor{block title}{fg=red,bg=white} % Change the block title color

\begin{block}{Acknowledgements}
\begin{itemize}
\item \small{\rmfamily{This project is under supervision of Dr.Bruno C. d. S. Oliveira and supportted by The University of Hong Kong Programming Languages Group.}}
\item \small{\rmfamily{Special thanks to Jeremy Bi, George Shi, Tomas, Emma, Ningning Xie, Linus Yang, Wexin Zhang and Yanlin.}} \\
\end{itemize}
\end{block}

%----------------------------------------------------------------------------------------
%	CONTACT INFORMATION
%----------------------------------------------------------------------------------------

%\setbeamercolor{block alerted title}{fg=black,bg=norange} % Change the alert block title colors
%\setbeamercolor{block alerted body}{fg=black,bg=white} % Change the alert block body colors

\begin{alertblock}{Contact Information}

\begin{itemize}
\item Web: https://github.com/hkuplg
\item Email: xiafan@hku.hk
\end{itemize}

\end{alertblock}

\begin{center}
\begin{tabular}{ccc}
\includegraphics[width=0.8\linewidth]{img/hkulogo} & \hfill &
\end{tabular}
\end{center}

%----------------------------------------------------------------------------------------

\end{column} % End of the third column

\end{columns} % End of all the columns in the poster

\end{frame} % End of the enclosing frame

\end{document}
