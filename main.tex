%%%%%%%%%%%%%%%%%%%%%%%%%%%%%%%%%%%%%%%%%
% Jacobs Landscape Poster
% LaTeX Template
% Version 1.1 (14/06/14)
%
% Created by:
% Computational Physics and Biophysics Group, Jacobs University
% https://teamwork.jacobs-university.de:8443/confluence/display/CoPandBiG/LaTeX+Poster
%
% Further modified by:
% Nathaniel Johnston (nathaniel@njohnston.ca)
%
% This template has been downloaded from:
% http://www.LaTeXTemplates.com
%
% License:
% CC BY-NC-SA 3.0 (http://creativecommons.org/licenses/by-nc-sa/3.0/)
%
%%%%%%%%%%%%%%%%%%%%%%%%%%%%%%%%%%%%%%%%%

%----------------------------------------------------------------------------------------
%	PACKAGES AND OTHER DOCUMENT CONFIGURATIONS
%----------------------------------------------------------------------------------------

\documentclass[final]{beamer}

\usepackage[scale=1.12]{beamerposter} % Use the beamerposter package for laying out the poster
\usepackage{listings}

\usetheme{confposter} % Use the confposter theme supplied with this template

\setbeamercolor{block title}{fg=ngreen,bg=white} % Colors of the block titles
\setbeamercolor{block body}{fg=black,bg=white} % Colors of the body of blocks
\setbeamercolor{block alerted title}{fg=white,bg=dblue!70} % Colors of the highlighted block titles
\setbeamercolor{block alerted body}{fg=black,bg=dblue!10} % Colors of the body of highlighted blocks
% Many more colors are available for use in beamerthemeconfposter.sty

\setbeamerfont{block body}{series=\rmfamily}

%-----------------------------------------------------------
% Define the column widths and overall poster size
% To set effective sepwid, onecolwid and twocolwid values, first choose how many columns you want and how much separation you want between columns
% In this template, the separation width chosen is 0.024 of the paper width and a 4-column layout
% onecolwid should therefore be (1-(# of columns+1)*sepwid)/# of columns e.g. (1-(4+1)*0.024)/4 = 0.22
% Set twocolwid to be (2*onecolwid)+sepwid = 0.464
% Set threecolwid to be (3*onecolwid)+2*sepwid = 0.708

\newlength{\sepwid}
\newlength{\onecolwid}
\newlength{\twocolwid}
\newlength{\threecolwid}
\setlength{\paperwidth}{36in} % A0 width: 46.8in
\setlength{\paperheight}{36in} % A0 height: 33.1in
\setlength{\sepwid}{0.024\paperwidth} % Separation width (white space) between columns
\setlength{\onecolwid}{0.22\paperwidth} % Width of one column
\setlength{\twocolwid}{0.464\paperwidth} % Width of two columns
\setlength{\threecolwid}{0.708\paperwidth} % Width of three columns
\setlength{\topmargin}{-0.5in} % Reduce the top margin size
%-----------------------------------------------------------

\usepackage{graphicx}  % Required for including images

\usepackage{booktabs} % Top and bottom rules for tables

%----------------------------------------------------------------------------------------
%	TITLE SECTION
%----------------------------------------------------------------------------------------

\title{Efficient Parser and Pretty Printer Combinators in F2J} % Poster title

\author{Yuteng Zhong, Yi Li and Fan Xia} % Author(s)

\institute{Department of Computer Science, The University of Hong Kong} % Institution(s)

%----------------------------------------------------------------------------------------

\begin{document}

\addtobeamertemplate{block end}{}{\vspace*{2ex}} % White space under blocks
\addtobeamertemplate{block alerted end}{}{\vspace*{2ex}} % White space under highlighted (alert) blocks

\setlength{\belowcaptionskip}{1ex} % White space under figures
\setlength\belowdisplayshortskip{2ex} % White space under equations

\renewcommand{\raggedright}{\leftskip=0pt \rightskip=0pt plus 0cm}

\begin{frame}[t] % The whole poster is enclosed in one beamer frame

\begin{columns}[t] % The whole poster consists of three major columns, the second of which is split into two columns twice - the [t] option aligns each column's content to the top

\begin{column}{\sepwid}\end{column} % Empty spacer column

\begin{column}{\onecolwid} % The first column

%----------------------------------------------------------------------------------------
%	The Tower of Babel
%----------------------------------------------------------------------------------------

\begin{figure}
\includegraphics[width=\linewidth]{img/babellong.png}
\caption{The Tower of Babel}
\end{figure}

%----------------------------------------------------------------------------------------
%	MOTIVATION
%----------------------------------------------------------------------------------------

\begin{block}{Motivation}

At HKU, the Programming Languages group is developing a new JVM-based compiler for functional languages called F2J. The overall goal of this project is to improve support for functional languages on the JVM, as currently existing compilation techniques have several limitations that limit the use of functional programming.

\begin{figure}
\includegraphics[width=0.9\linewidth]{img/bootstrapping.jpg}
\caption{Bootstrapping}
\end{figure}

In order to make a selfhosting compiler, in other word, a \textbf{bootstrapped compiler} for F2J, we will need a parser combinators library. In this process, we could also build a combinators library for pretty printing with a similar approach, which is a reversed process of parsing.

\end{block}

%----------------------------------------------------------------------------------------
%	INTRODUCTION
%----------------------------------------------------------------------------------------



\end{column} % End of the first column

\begin{column}{\sepwid}\end{column} % Empty spacer column

\begin{column}{\twocolwid} % Begin a column which is two columns wide (column 2)

\begin{columns}[t,totalwidth=\twocolwid] % Split up the two columns wide column

\begin{column}{\onecolwid}\vspace{-.6in} % The first column within column 2 (column 2.1)

%----------------------------------------------------------------------------------------
%	MATERIALS
%----------------------------------------------------------------------------------------

\begin{block}{Introduction}

Parser combinators are a set of higher-order functions that accepts several parser as input and then return a parser as output. In this context, a parser is a function accepting strings as input and returning some structure as output, typically a parse tree or a set of indices representing locations in the string where parsing stopped successfully. Parser combinators enable a recursive descent parsing strategy that facilitates modular piecewise construction and testing.
This parsing technique is called combinatory parsing. Parser and Pretty Printer combinators are an alternative to tools used in compiler constructions, such as lex, yacc or antlr. They have the advantage of being a library instead of a code generator tool.

\end{block}

%----------------------------------------------------------------------------------------

\end{column} % End of column 2.1

\begin{column}{\onecolwid}\vspace{-.6in} % The second column within column 2 (column 2.2)

%----------------------------------------------------------------------------------------
%	METHODS
%----------------------------------------------------------------------------------------

\begin{block}{Methods}

\textbf{Algebra of programming:}

\textbf{Monadic:}

Monad is an algebraic structure from machematics that provided useful for addressing a
number of computational problems. Using a monadic sequencing combinator of parsers avoids
the messy manipulation of nested tuples of results present in early work. Moreover, using
monad comprehension notation makes parsers more compat and easy to read.

% The monad of parser can be expressed in a modular way in terms of two simpler monads. The
% immediate benefit is that the basic parser combinators no longer needs to be defined
% explicitly. Rather, they arise automatically as a special case of lifting monad operations
% from the base monad \textit{m} to a certain other monad parameterised over \textit{m}. This
% also means that, if we change the nature of parsers by modifying the base monad, then new
% combinators for the modified monad of parsers also arise automically via the lifting
% construction.


\end{block}

%----------------------------------------------------------------------------------------

\end{column} % End of column 2.2

\end{columns} % End of the split of column 2 - any content after this will now take up 2 columns width

%----------------------------------------------------------------------------------------
%	IMPORTANT RESULT
%----------------------------------------------------------------------------------------
\begin{alertblock}{monad}

\begin{figure}
\includegraphics[width=0.8\linewidth]{img/monad.jpg}

\end{figure}
\end{alertblock}

%----------------------------------------------------------------------------------------

\begin{columns}[t,totalwidth=\twocolwid] % Split up the two columns wide column again

\begin{column}{\onecolwid} % The first column within column 2 (column 2.1)

%----------------------------------------------------------------------------------------
%	MATHEMATICAL SECTION
%----------------------------------------------------------------------------------------

\begin{block}{Optimisation}

\textbf{Parser}
\begin{itemize}
\item Lazy evaluation. Parser combinators will generates all possible results in each level
of parse tree, lazy evaluation can enable the possibilty of ''parse by need''.
%, which will save lots of memory and time and become much more efficient.
\item Reduce backtracking. Backtracking is required when the parser encounter a failure.
% and then it may come back to the begining and try the other alternative rules.
\item More specific rules. Try the most possible rules based on parsing context will help to find the correct result earlier.
%The parser will try the rules one by one.
\end{itemize}

\end{block}

%----------------------------------------------------------------------------------------

\end{column} % End of column 2.1

\begin{column}{\onecolwid} % The second column within column 2 (column 2.2)

%----------------------------------------------------------------------------------------
%	RESULTS
%----------------------------------------------------------------------------------------

\begin{block}{Optimisation}



\textbf{Pretty Printer}
\begin{itemize}
\item Algebra: Everything is based on a single concatenation operator that is associative.
\item Expressiveness: Not as expressive as Hugh's library, but it is enough for using.
\item Optimality: Optimal and Bounded. It means that the lib can choose line breaks to avoid overflow whenever possible, and it's done in O(n).
\end{itemize}




%\begin{table}
%\vspace{2ex}
%\begin{tabular}{l l l}
%\toprule
%\textbf{Treatments} & \textbf{Response 1} & \textbf{Response 2}\\
%\midrule
%Treatment 1 & 0.0003262 & 0.562 \\
%Treatment 2 & 0.0015681 & 0.910 \\
%Treatment 3 & 0.0009271 & 0.296 \\
%\bottomrule
%\end{tabular}
%\caption{Table caption}
%\end{table}

\end{block}

%----------------------------------------------------------------------------------------

\end{column} % End of column 2.2

\end{columns} % End of the split of column 2

\end{column} % End of the second column

\begin{column}{\sepwid}\end{column} % Empty spacer column

\begin{column}{\onecolwid} % The third column


\begin{block}{Case Study}

\textbf{ArithExpr}
% \begin{lstlisting}[frame=single]
% expr        ::= expr operations factor | factor
% operations  ::= + | - | * | /
% factor      ::= nat | ( expr )
% \end{lstlisting}
\textbf{XML}
\textbf{F2J}
\textbf{FWJava}

\end{block}

%----------------------------------------------------------------------------------------
%	CONCLUSION
%----------------------------------------------------------------------------------------


\begin{block}{Conclusion}

In the world of Pretty Printer, Richard Bird put algebraic design on the map. John Hughes made pretty printer libraries a prime example of the algebraic approach. Then Philip Wadler made a lot improvement to Hughes's lib. The greatest homage is imitation, and here we have paid as much homage as we can to Hughes, Bird and Wadler.


\end{block}

%----------------------------------------------------------------------------------------
%	REFERENCES
%----------------------------------------------------------------------------------------

\begin{block}{References}

\nocite{*} % Insert publications even if they are not cited in the poster
\small{\bibliographystyle{unsrt}
\bibliography{sample}\vspace{0.75in}}

\end{block}

%----------------------------------------------------------------------------------------
%	ACKNOWLEDGEMENTS
%----------------------------------------------------------------------------------------

\setbeamercolor{block title}{fg=red,bg=white} % Change the block title color

\begin{block}{Acknowledgements}

\small{\rmfamily{Nam mollis tristique neque eu luctus. Suspendisse rutrum congue nisi sed convallis. Aenean id neque dolor. Pellentesque habitant morbi tristique senectus et netus et malesuada fames ac turpis egestas.}} \\

\end{block}

%----------------------------------------------------------------------------------------
%	CONTACT INFORMATION
%----------------------------------------------------------------------------------------

\setbeamercolor{block alerted title}{fg=black,bg=norange} % Change the alert block title colors
\setbeamercolor{block alerted body}{fg=black,bg=white} % Change the alert block body colors

\begin{alertblock}{Contact Information}

\begin{itemize}
\item Web: \href{http://www.university.edu/smithlab}{http://www.university.edu/smithlab}
\item Email: \href{mailto:john@smith.com}{john@smith.com}
\item Phone: +1 (000) 111 1111
\end{itemize}

\end{alertblock}

\begin{center}
\begin{tabular}{ccc}
\includegraphics[width=0.4\linewidth]{logo.png} & \hfill & \includegraphics[width=0.4\linewidth]{logo.png}
\end{tabular}
\end{center}

%----------------------------------------------------------------------------------------

\end{column} % End of the third column

\end{columns} % End of all the columns in the poster

\end{frame} % End of the enclosing frame

\end{document}
